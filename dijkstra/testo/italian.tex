\usepackage{xcolor}
\usepackage{afterpage}
\usepackage{pifont,mdframed}
\usepackage[bottom]{footmisc}

\usepackage{amsmath}
\usepackage{amsthm}
\usepackage{amssymb}
\usepackage{mathtools}


\createsection{\Grader}{Grader di prova}

\newcommand{\inputfile}{\texttt{stdin}}
\newcommand{\outputfile}{\texttt{stdout}}

\newenvironment{warning}
  {\par\begin{mdframed}[linewidth=2pt,linecolor=gray]%
    \begin{list}{}{\leftmargin=1cm
                   \labelwidth=\leftmargin}\item[\Large\ding{43}]}
  {\end{list}\end{mdframed}\par}

% % % % % % % % % % % % % % % % % % % % % % % % % % % % % % % % % % % % % % % % % % %
% % % % % % % % % % % % % % % % % % % % % % % % % % % % % % % % % % % % % % % % % % %

Trovare la distanza fra il vertice $0$ e il vertice $N-1$ in un grafo orientato pesato.

%       Risolvete Dijkstra su un grafo diretto.

% % % % % % % % % % % % % % % % % % % % % % % % % % % % % % % % % % % % % % % % % % %
% % % % % % % % % % % % % % % % % % % % % % % % % % % % % % % % % % % % % % % % % % %

\begin{itemize}[nolistsep]
  \item L'intero $N$ rappresenta il numero di vertici.
  \item L'intero $M$ rappresenta il numero di archi.
  \item Gli array \texttt{X}, \texttt{Y} e \texttt{P} sono indicizzati da $0$ a $M-1$ e indicano che esiste un arco dal vertice \texttt{X[$i$]} al vertice \texttt{Y[$i$]} di peso \texttt{P[$i$]}.
  \item La funzione deve restituire la minima distanza fra il verice $0$ e il vertice $N-1$; se non è possibile raggiungere il vertice $N-1$, deve restituire il valore $-1$.
\end{itemize}

\medskip

% % % % % % % % % % % % % % % % % % % % % % % % % % % % % % % % % % % % % % % % % % %
% % % % % % % % % % % % % % % % % % % % % % % % % % % % % % % % % % % % % % % % % % %


\Grader
Su input ci sono $M+1$ righe, contenenti:
\begin{itemize}[nolistsep,itemsep=2mm]
\item Riga $1$: i due interi $N$ e $M$, il numero di vertici e di archi.
\item Riga $2\dots M+1$: ciascuna riga contiene $3$ numeri, $a_i$, $b_i$ e $p_i$, che indicano che esiste un arco di peso $p_i$ che connette $a_i$ a $b_i$.

\end{itemize}

Il file di output è composto da un'unica riga, contenente la distanza minima. Se non esiste alcun cammino che collega $0$ a $N-1$, stampare $-1$.

% % % % % % % % % % % % % % % % % % % % % % % % % % % % % % % % % % % % % % % % % % %
% % % % % % % % % % % % % % % % % % % % % % % % % % % % % % % % % % % % % % % % % % %


\Constraints

\begin{itemize}[nolistsep, itemsep=2mm]
	\item $1 \le N \le 200\,000$.
    \item $1 \le M \le 200\,000$.
	\item $p_i \le 1\,000$ per ogni $i=0,\ldots, M-1$.
\end{itemize}

% % % % % % % % % % % % % % % % % % % % % % % % % % % % % % % % % % % % % % % % % % %
% % % % % % % % % % % % % % % % % % % % % % % % % % % % % % % % % % % % % % % % % % %


\Examples

\begin{example}
\exmpfile{mincammino.input0.txt}{mincammino.output0.txt}%
\exmpfile{mincammino.input1.txt}{mincammino.output1.txt}%
\end{example}

% % % % % % % % % % % % % % % % % % % % % % % % % % % % % % % % % % % % % % % % % % %
% % % % % % % % % % % % % % % % % % % % % % % % % % % % % % % % % % % % % % % % % % %


% \Explanation
% 
% Il \textbf{primo caso di esempio} è quello descritto nel testo.\\[2mm]
% Nel \textbf{secondo caso di esempio} conviene fermarsi nelle spiagge 2, 3, 4, 6 e 7 (ma è possibile fermarsi anche nella spiaggia 1 ottenendo il medesimo risultato).

%%%%%%%%%%%%%%%%%%%%%%%%%%%%%%%%%%%%
%%%%%%%%%%%%%%%%%%%%%%%%%%%%%%%%%%%%
