\usepackage{xcolor}
\usepackage{afterpage}
\usepackage{pifont,mdframed}
\usepackage[bottom]{footmisc}

\createsection{\Grader}{Grader di prova}

\renewcommand{\inputfile}{\texttt{stdin}}
\renewcommand{\outputfile}{\texttt{stdout}}
\makeatletter
\renewcommand{\this@inputfilename}{\texttt{stdin}}
\renewcommand{\this@outputfilename}{\texttt{stdout}}
\makeatother


\newenvironment{warning}
  {\par\begin{mdframed}[linewidth=2pt,linecolor=gray]%
    \begin{list}{}{\leftmargin=1cm
                   \labelwidth=\leftmargin}\item[\Large\ding{43}]}
  {\end{list}\end{mdframed}\par}

% % % % % % % % % % % % % % % % % % % % % % % % % % % % % % % % % % % % % % % % % % %
% % % % % % % % % % % % % % % % % % % % % % % % % % % % % % % % % % % % % % % % % % %

Costruire un albero binario di ricerca le cui chiavi sono gli elementi
dell'array (già ordinato) in input e effettuare le operazioni (ricerca,
inserimento, cancellazione, numero di elementi in un range) specificate.

% % % % % % % % % % % % % % % % % % % % % % % % % % % % % % % % % % % % % % % % % % %
% % % % % % % % % % % % % % % % % % % % % % % % % % % % % % % % % % % % % % % % % % %


\InputFile
Su standard input (\texttt{scanf}) sono presenti $N+O+1$ righe, contenenti:
\begin{itemize}[nolistsep,itemsep=2mm]
\item Riga $1$: gli interi $N$ e $O$, la dimensione dell'array e il numero di
    operazioni.
\item Righe $2...N+1$: gli $N$ elementi $a_i$ dell'array.
\item Righe $N+2...O+N+1$: Una delle seguenti possibilità:
    \begin{itemize}
        \item il carattere \texttt{f}, seguito da uno spazio e da
        un numero $o_i$, a indicare la richiesta di ricercare il numero $o_i$
        nell'albero
        \item il carattere \texttt{i}, seguito da uno spazio e da
        un numero $o_i$, a indicare la richiesta di inserire il numero $o_i$
        nell'albero
        \item il carattere \texttt{d}, seguito da uno spazio e da
        un numero $o_i$, a indicare la richiesta di eliminare il numero $o_i$
        dall'albero
        \item il carattere \texttt{r}, seguito da uno spazio e da
        due numeri $s_i$, $e_i$ a indicare la richiesta di contare il numero
        di elementi nell'albero che appartengono al range $[s_i, e_i)$.
    \end{itemize}
\end{itemize}

\OutputFile
Su standard output (\texttt{printf}) per ogni richiesta di ricerca di un
numero deve essere stampato su una riga $1$ se il numero è stato trovato, $0$
altrimenti. Per ogni richiesta di ricerca di un range deve essere stampato il
numero di elementi facenti parte di quel range.
	
% % % % % % % % % % % % % % % % % % % % % % % % % % % % % % % % % % % % % % % % % % %
% % % % % % % % % % % % % % % % % % % % % % % % % % % % % % % % % % % % % % % % % % %


\Constraints

\begin{itemize}[nolistsep, itemsep=2mm]
    \item $0 \leq N \leq 100\,000$.
    \item $1 \leq O \leq 400\,000$.
    \item $a_i < a_{i+1}$ per ogni $i$.
    \item I primi cinque casi di test sono quelli riportati sotto.
    \item I cinque casi di test successivi presentano solo operazioni di
        ricerca.
    \item I cinque casi di test successivi presentano solo operazioni di
        ricerca e di inserimento.
    \item I cinque casi di test successivi non presentano operazioni di
        ricerca di un range.
    \item I cinque casi di test successivi non presentano operazioni di
        cancellazione.
    \item Gli ultimi dieci casi di test presentano tutte le operazioni
        possibili.
\end{itemize}

% % % % % % % % % % % % % % % % % % % % % % % % % % % % % % % % % % % % % % % % % % %
% % % % % % % % % % % % % % % % % % % % % % % % % % % % % % % % % % % % % % % % % % %


\Examples

\begin{example}
\exmpfile{bst.input0.txt}{bst.output0.txt}%
\end{example}
\begin{example}
\exmpfile{bst.input1.txt}{bst.output1.txt}%
\end{example}
\begin{example}
\exmpfile{bst.input2.txt}{bst.output2.txt}%
\end{example}
\begin{example}
\exmpfile{bst.input3.txt}{bst.output3.txt}%
\end{example}
\begin{example}
\exmpfile{bst.input4.txt}{bst.output4.txt}%
\end{example}

% % % % % % % % % % % % % % % % % % % % % % % % % % % % % % % % % % % % % % % % % % %
% % % % % % % % % % % % % % % % % % % % % % % % % % % % % % % % % % % % % % % % % % %


%\Explanation
