\usepackage{xcolor}
\usepackage{afterpage}
\usepackage{pifont,mdframed}
\usepackage[bottom]{footmisc}
\usepackage{minted}

\createsection{\Grader}{Grader di prova}

\renewcommand{\inputfile}{\texttt{stdin}}
\renewcommand{\outputfile}{\texttt{stdout}}
\makeatletter
\renewcommand{\this@inputfilename}{\texttt{stdin}}
\renewcommand{\this@outputfilename}{\texttt{stdout}}
\makeatother


\newenvironment{warning}
  {\par\begin{mdframed}[linewidth=2pt,linecolor=gray]%
    \begin{list}{}{\leftmargin=1cm
                   \labelwidth=\leftmargin}\item[\Large\ding{43}]}
  {\end{list}\end{mdframed}\par}

% % % % % % % % % % % % % % % % % % % % % % % % % % % % % % % % % % % % % % % % % % %
% % % % % % % % % % % % % % % % % % % % % % % % % % % % % % % % % % % % % % % % % % %

Stampare l'intero che si troverebbe nella posizione $K$ dell'array $A$ fornito in
input se $A$ fosse ordinato in ordine crescente.

\begin{warning}
L'input di questo problema è molto grande. Per non superare il tempo limite con
la sola lettura dell'input, si consiglia di usare la seguente funzione al posto di
\texttt{scanf} o \texttt{cout}:
  \begin{verbatim}
    int nextInt() {
      int n = 0;
      bool negative = false;
      int c = getchar();
      while ((c < '0' || c > '9') && c != '-')
        c = getchar();
      if (c == '-') {
        negative = true;
        c = getchar();
      }
      while (c >= '0' && c <= '9') {
        n = 10 * n + c - '0';
        c = getchar();
      }
      return negative ? -n : n;
    }
  \end{verbatim}
\end{warning}

% % % % % % % % % % % % % % % % % % % % % % % % % % % % % % % % % % % % % % % % % % %
% % % % % % % % % % % % % % % % % % % % % % % % % % % % % % % % % % % % % % % % % % %


\InputFile
Su standard input (\texttt{scanf}) sono presenti due righe, contenenti:
\begin{itemize}[nolistsep,itemsep=2mm]
\item Riga $1$: l'intero $N$, la dimensione dell'array da ordinare, e l'intero $K$, la posizione da stampare.
\item Riga $2$: gli $N$ interi $a_i$ da ordinare.
\end{itemize}

\OutputFile
Su standard output (\texttt{printf}) deve essere scritta esattamente una riga,
contenente $1$ numero, ovvero l'elemento che si troverebbe in posizione $K$ se l'array fosse ordinato.
	
% % % % % % % % % % % % % % % % % % % % % % % % % % % % % % % % % % % % % % % % % % %
% % % % % % % % % % % % % % % % % % % % % % % % % % % % % % % % % % % % % % % % % % %


\Constraints

\begin{itemize}[nolistsep, itemsep=2mm]
    \item $1 \leq N \leq 5\,000\,000$.
    \item $-1\,000\,000 \leq a_i \leq 1\,000\,000$ per ogni $i$.
\end{itemize}

% % % % % % % % % % % % % % % % % % % % % % % % % % % % % % % % % % % % % % % % % % %
% % % % % % % % % % % % % % % % % % % % % % % % % % % % % % % % % % % % % % % % % % %


\Examples

\begin{example}
\exmpfile{sort.input0.txt}{sort.output0.txt}%
\end{example}

% % % % % % % % % % % % % % % % % % % % % % % % % % % % % % % % % % % % % % % % % % %
% % % % % % % % % % % % % % % % % % % % % % % % % % % % % % % % % % % % % % % % % % %


%\Explanation
