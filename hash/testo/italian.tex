\usepackage{xcolor}
\usepackage{afterpage}
\usepackage{pifont,mdframed}
\usepackage[bottom]{footmisc}

\createsection{\Grader}{Grader di prova}

\renewcommand{\inputfile}{\texttt{stdin}}
\renewcommand{\outputfile}{\texttt{stdout}}
\makeatletter
\renewcommand{\this@inputfilename}{\texttt{stdin}}
\renewcommand{\this@outputfilename}{\texttt{stdout}}
\makeatother


\newenvironment{warning}
  {\par\begin{mdframed}[linewidth=2pt,linecolor=gray]%
    \begin{list}{}{\leftmargin=1cm
                   \labelwidth=\leftmargin}\item[\Large\ding{43}]}
  {\end{list}\end{mdframed}\par}

% % % % % % % % % % % % % % % % % % % % % % % % % % % % % % % % % % % % % % % % % % %
% % % % % % % % % % % % % % % % % % % % % % % % % % % % % % % % % % % % % % % % % % %

Costruire una tabella hash vuota su cui effettuare le operazioni (ricerca,
inserimento, cancellazione) specificate.

% % % % % % % % % % % % % % % % % % % % % % % % % % % % % % % % % % % % % % % % % % %
% % % % % % % % % % % % % % % % % % % % % % % % % % % % % % % % % % % % % % % % % % %


\InputFile
Su standard input (\texttt{scanf}) sono presenti $O+1$ righe, contenenti:
\begin{itemize}[nolistsep,itemsep=2mm]
\item Riga $1$: l'intero $O$, il numero di operazioni.
\item Righe $2...O+1$: Una delle seguenti possibilità:
    \begin{itemize}
        \item il carattere \texttt{f}, seguito da uno spazio e da
        un numero $o_i$, a indicare la richiesta di ricercare il numero $o_i$
        \item il carattere \texttt{i}, seguito da uno spazio e da
        un numero $o_i$, a indicare la richiesta di inserire il numero $o_i$
        \item il carattere \texttt{d}, seguito da uno spazio e da
        un numero $o_i$, a indicare la richiesta di eliminare il numero $o_i$
    \end{itemize}
\end{itemize}

\OutputFile
Su standard output (\texttt{printf}) per ogni richiesta di ricerca di un
numero deve essere stampato su una riga $1$ se il numero è stato trovato, $0$
altrimenti.
	
% % % % % % % % % % % % % % % % % % % % % % % % % % % % % % % % % % % % % % % % % % %
% % % % % % % % % % % % % % % % % % % % % % % % % % % % % % % % % % % % % % % % % % %


\Constraints

\begin{itemize}[nolistsep, itemsep=2mm]
    \item $1 \leq O \leq 1\,000\,000$.
    \item I primi due casi di test sono quelli riportati sotto.
    \item I dieci casi di test successivi presentano solo operazioni di
        ricerca e di inserimento.
    \item Gli ultimi dieci casi di test presentano tutte le operazioni
        possibili.
\end{itemize}

% % % % % % % % % % % % % % % % % % % % % % % % % % % % % % % % % % % % % % % % % % %
% % % % % % % % % % % % % % % % % % % % % % % % % % % % % % % % % % % % % % % % % % %


\Examples

\begin{example}
\exmpfile{hash.input0.txt}{hash.output0.txt}%
\end{example}
\begin{example}
\exmpfile{hash.input1.txt}{hash.output1.txt}%
\end{example}

% % % % % % % % % % % % % % % % % % % % % % % % % % % % % % % % % % % % % % % % % % %
% % % % % % % % % % % % % % % % % % % % % % % % % % % % % % % % % % % % % % % % % % %


%\Explanation
