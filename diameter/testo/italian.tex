\usepackage{xcolor}
\usepackage{afterpage}
\usepackage{pifont,mdframed}
\usepackage[bottom]{footmisc}

\createsection{\Grader}{Grader di prova}

\renewcommand{\inputfile}{\texttt{stdin}}
\renewcommand{\outputfile}{\texttt{stdout}}
\makeatletter
\renewcommand{\this@inputfilename}{\texttt{stdin}}
\renewcommand{\this@outputfilename}{\texttt{stdout}}
\makeatother


\newenvironment{warning}
  {\par\begin{mdframed}[linewidth=2pt,linecolor=gray]%
    \begin{list}{}{\leftmargin=1cm
                   \labelwidth=\leftmargin}\item[\Large\ding{43}]}
  {\end{list}\end{mdframed}\par}

% % % % % % % % % % % % % % % % % % % % % % % % % % % % % % % % % % % % % % % % % % %
% % % % % % % % % % % % % % % % % % % % % % % % % % % % % % % % % % % % % % % % % % %

  Dato un grafo connesso in input, calcolarne il diametro.

\begin{warning}
    Il \textit{diametro} di un grafo è definito come la massima lunghezza di un
    cammino minimo tra una qualsiasi coppia di nodi.
\end{warning}

% % % % % % % % % % % % % % % % % % % % % % % % % % % % % % % % % % % % % % % % % % %
% % % % % % % % % % % % % % % % % % % % % % % % % % % % % % % % % % % % % % % % % % %


\InputFile
Su standard input (\texttt{scanf}) sono presenti $M+1$ righe, contenenti:
\begin{itemize}[nolistsep,itemsep=2mm]
\item Riga $1$: gli interi $N$ e $M$, il numero di nodi e di archi, rispettivamente.
\item Riga $2\dots M+1$: ciascuna riga contiene $2$ numeri, $a_i$ e $b_i$,
che indicano che esiste un arco che connette tra loro i nodi $a_i$ e $b_i$.
\end{itemize}
\begin{warning} 
    I nodi sono numerati da $0$ a $N-1$.
\end{warning}

\OutputFile
Su standard output (\texttt{printf}) deve essere scritta esattamente $1$ riga,
contenente un singolo numero, ovvero il diametro del grafo.
	
% % % % % % % % % % % % % % % % % % % % % % % % % % % % % % % % % % % % % % % % % % %
% % % % % % % % % % % % % % % % % % % % % % % % % % % % % % % % % % % % % % % % % % %


\Constraints

\begin{itemize}[nolistsep, itemsep=2mm]
    \item $1 \leq N \leq 1\,000$.
    \item $1 \leq M \leq 10\,000$.
\end{itemize}

% % % % % % % % % % % % % % % % % % % % % % % % % % % % % % % % % % % % % % % % % % %
% % % % % % % % % % % % % % % % % % % % % % % % % % % % % % % % % % % % % % % % % % %


\Examples

\begin{example}
    \exmpfile{diameter.input0.txt}{diameter.output0.txt}%
\end{example}

% % % % % % % % % % % % % % % % % % % % % % % % % % % % % % % % % % % % % % % % % % %
% % % % % % % % % % % % % % % % % % % % % % % % % % % % % % % % % % % % % % % % % % %


%\Explanation
