\usepackage{xcolor}
\usepackage{afterpage}
\usepackage{pifont,mdframed}
\usepackage[bottom]{footmisc}

\createsection{\Grader}{Grader di prova}

\renewcommand{\inputfile}{\texttt{stdin}}
\renewcommand{\outputfile}{\texttt{stdout}}
\makeatletter
\renewcommand{\this@inputfilename}{\texttt{stdin}}
\renewcommand{\this@outputfilename}{\texttt{stdout}}
\makeatother


\newenvironment{warning}
  {\par\begin{mdframed}[linewidth=2pt,linecolor=gray]%
    \begin{list}{}{\leftmargin=1cm
                   \labelwidth=\leftmargin}\item[\Large\ding{43}]}
  {\end{list}\end{mdframed}\par}

% % % % % % % % % % % % % % % % % % % % % % % % % % % % % % % % % % % % % % % % % % %
% % % % % % % % % % % % % % % % % % % % % % % % % % % % % % % % % % % % % % % % % % %

Calcola l'altezza dell'albero binario in input.

% % % % % % % % % % % % % % % % % % % % % % % % % % % % % % % % % % % % % % % % % % %
% % % % % % % % % % % % % % % % % % % % % % % % % % % % % % % % % % % % % % % % % % %


\InputFile
Su standard input (\texttt{scanf}) sono presenti $N+1$ righe, contenenti:
\begin{itemize}[nolistsep,itemsep=2mm]
\item Riga $1$: l'intero $N$, la dimensione dell'albero, e l'intero $r$,
l'\textit{id} della radice dell'albero.
\item Riga $2\dots N+1$: ciascuna riga contiene $3$ numeri, $id_i$, $l_i$, $r_i$,
che indicano che il nodo con id $id_i$ ha per figli, rispettivamente,
sinistro e destro i nodi con id $l_i$ e $r_i$. Se $l_i = -1$ significa
che il nodo $i$ non ha figli sinistri, analogamente per $r_i$.
\end{itemize}

\OutputFile
Su standard output (\texttt{printf}) deve essere scritta esattamente una riga,
contenente un singolo numero, ovvero l'altezza dell'albero in input.
	
% % % % % % % % % % % % % % % % % % % % % % % % % % % % % % % % % % % % % % % % % % %
% % % % % % % % % % % % % % % % % % % % % % % % % % % % % % % % % % % % % % % % % % %


\Constraints

\begin{itemize}[nolistsep, itemsep=2mm]
    \item $1 \leq N \leq 1\,000\,000$.
    \item $0 \leq r \leq N-1$.
    \item $0 \leq id_i \leq N-1$ per ogni $i$.
    \item $0 \leq l_i \leq N-1$ per ogni $i$.
    \item $0 \leq r_i \leq N-1$ per ogni $i$.
    \item L'input rappresenta correttamente un albero.
\end{itemize}

% % % % % % % % % % % % % % % % % % % % % % % % % % % % % % % % % % % % % % % % % % %
% % % % % % % % % % % % % % % % % % % % % % % % % % % % % % % % % % % % % % % % % % %


\Examples

\begin{example}
\exmpfile{height.input0.txt}{height.output0.txt}%
\end{example}

% % % % % % % % % % % % % % % % % % % % % % % % % % % % % % % % % % % % % % % % % % %
% % % % % % % % % % % % % % % % % % % % % % % % % % % % % % % % % % % % % % % % % % %


%\Explanation
