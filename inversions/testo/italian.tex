\usepackage{xcolor}
\usepackage{afterpage}
\usepackage{pifont,mdframed}
\usepackage[bottom]{footmisc}

\createsection{\Grader}{Grader di prova}

\renewcommand{\inputfile}{\texttt{stdin}}
\renewcommand{\outputfile}{\texttt{stdout}}
\makeatletter
\renewcommand{\this@inputfilename}{\texttt{stdin}}
\renewcommand{\this@outputfilename}{\texttt{stdout}}
\makeatother


\newenvironment{warning}
  {\par\begin{mdframed}[linewidth=2pt,linecolor=gray]%
    \begin{list}{}{\leftmargin=1cm
                   \labelwidth=\leftmargin}\item[\Large\ding{43}]}
  {\end{list}\end{mdframed}\par}

% % % % % % % % % % % % % % % % % % % % % % % % % % % % % % % % % % % % % % % % % % %
% % % % % % % % % % % % % % % % % % % % % % % % % % % % % % % % % % % % % % % % % % %

    Conta il numero di coppie $i < j$ per cui $A[i] > A[j]$, dove $A$ è l'array in input.
    $A$ non contiene valori duplicati.

% % % % % % % % % % % % % % % % % % % % % % % % % % % % % % % % % % % % % % % % % % %
% % % % % % % % % % % % % % % % % % % % % % % % % % % % % % % % % % % % % % % % % % %


\InputFile
Su standard input (\texttt{scanf}) sono presenti due righe, contenenti:
\begin{itemize}[nolistsep,itemsep=2mm]
\item Riga $1$: l'intero $N$, la dimensione dell'array da ordinare
\item Riga $2$: gli $N$ interi $a_i$ da ordinare.
\end{itemize}

\OutputFile
Su standard output (\texttt{printf}) deve essere scritta esattamente una riga,
contenente $1$ numero, che rappresenta il numero di inversioni presenti nell'array.

\begin{warning}
  Questo numero potrebbe essere molto grande (più di $2^{31}$), quindi si consiglia
    di usare un intero a $64$ bit per contenerlo (\texttt{long long}).
\end{warning}
	
% % % % % % % % % % % % % % % % % % % % % % % % % % % % % % % % % % % % % % % % % % %
% % % % % % % % % % % % % % % % % % % % % % % % % % % % % % % % % % % % % % % % % % %


\Constraints

\begin{itemize}[nolistsep, itemsep=2mm]
    \item $1 \leq N \leq 200\,000$.
    \item $-1\,000\,000 \leq a_i \leq 1\,000\,000$ per ogni $i$.
    \item $a_i \neq a_j$ se $i \neq j$.
\end{itemize}

% % % % % % % % % % % % % % % % % % % % % % % % % % % % % % % % % % % % % % % % % % %
% % % % % % % % % % % % % % % % % % % % % % % % % % % % % % % % % % % % % % % % % % %


\Examples

\begin{example}
\exmpfile{sort.input0.txt}{sort.output0.txt}%
\end{example}

% % % % % % % % % % % % % % % % % % % % % % % % % % % % % % % % % % % % % % % % % % %
% % % % % % % % % % % % % % % % % % % % % % % % % % % % % % % % % % % % % % % % % % %


%\Explanation
